
\chapter{Introduction}

%%%%%%%%%%%%%%%%%%%%%%%%%%%%%%%%%%%%%%%%%%%%%%%%%%%%%%%%%%%%%%%%%%%%%
\section{Overview}

LSSP is a {\textbf{\color{red}{L}}}inear {\textbf{\color{red}{S}}}olver 
library for {\textbf{\color{red}{SP}}}arse linear system, \verb|Ax = b|. 
The package is designed for Linux, Unix and Mac systems. 
It is also possible to compile under Windows. 
The code is written by C++ and C, and it is serial.

LSSP has implemented many solvers and preconditioners, and it also has interfaces to external packages, 
     such as PETSc, MUMPS, FASP, UMFPACK (SUITESPARSE), KLU (SUITESPARSE), LASPACK, 
     ITSOL, LIS, QR\_MUMPS, PARDISO, SUPERLU, and HSL MI20.

%%%%%%%%%%%%%%%%%%%%%%%%%%%%%%%%%%%%%%%%%%%%%%%%%%%%%%%%%%%%%%%%%%%%%
\section{Cite}
If you like our LSSP library, you may cite it like this,
\begin{evb}
@misc{lssp-library,
    author="Hui Liu",
    title="LSSP: a Linear Solver Package for Sparse Matrix",
    year="2019",
    note={\url{https://github.com/huiscliu/lssp/}}
}
\end{evb}


%%%%%%%%%%%%%%%%%%%%%%%%%%%%%%%%%%%%%%%%%%%%%%%%%%%%%%%%%%%%%%%%%%%%%
\section{Website}
The official website for LSSP is \url{https://github.com/huiscliu/lssp/}.


