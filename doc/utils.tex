\chapter{Utilities}

\section{Print}

\vb{lssp_set_log} sets log file handler.
\begin{evb}
void lssp_set_log(FILE *io);
\end{evb}

\vb{lssp_printf} outputs to stdout and default log file if set.
\begin{evb}
int lssp_printf(const char *fmt, ...);
\end{evb}

\vb{lssp_error} prints output error message and quits with error code.
\begin{evb}
void lssp_error(int code, const char *fmt, ...);
\end{evb}

\vb{lssp_warning} print warning info.
\begin{evb}
void lssp_warning(const char *fmt, ...);
\end{evb}

\section{Memory}
The following functions provide memory allocation, calloc, reallocation, freeing and copying.
\begin{evb}
template <typename T> T * lssp_malloc(const int n);

template <typename T> T * lssp_calloc(const int n);

template <typename T> T * lssp_realloc(T * old, const int n);

template <typename T> void lssp_free(T * &p);

template<typename T> void lssp_memcpy_on(T *dst, const T *src, const int n);

template <typename T> T * lssp_copy_on(const T *src, const int n);
\end{evb}

\section{Performance}

\vb{lssp_get_time} gets current time.
\begin{evb}
double lssp_get_time();
\end{evb}

\vb{lssp_get_mem_usage} returns current memory usage. If \vb{peak} is not \vb{NULL},
then peak memory is returned to \vb{peak}.
\begin{evb}
double lssp_get_mem_usage(double *peak);
\end{evb}
